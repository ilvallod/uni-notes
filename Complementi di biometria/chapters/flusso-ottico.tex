\section{Flusso ottico e impieghi in biometria}
Tecnica che consente di analizzare il movimento di oggetti all'interno di una scena attraverso l'elaborazione di immagini. Questa tecnica viene utilizzata in diverse applicazioni, tra cui la guida autonoma dei droni e dei robot, dove è necessario rilevare gli oggetti. Viene impiegato anche in ambito di biometria, per l'anti-spoofing e per la biometria comportamentale. 

\paragraph{Flusso ottico} 
E' una misura della direzione e della velocità del movimento dell'oggetto nella scena ripresa dall'immagine

\paragraph{Motion field}
Proiezione nell’immagine di vettori di movimento tridimensionali

\paragraph{Applicazione biometriche del flusso ottico}
\begin{itemize}
    \item Videosorveglianza
    \item Face tracking, posso analizzare micro e macro-espressioni
    \item Pattern comportamentale dalla camminata
\end{itemize}

\paragraph{Histogram of Oriented Optical Flow}
Tecnica di elaborazione dell'immagine utilizzata per analizzare il flusso ottico. Divide l'immagine in regioni e calcola il flusso ottico per ciascuna regione, quindi calcola l'orientazione di ogni vettore di flusso ottico e costruisce un istogramma delle orientazioni. Questo istogramma può quindi essere utilizzato come rappresentazione dell'informazione del flusso ottico

\newpage