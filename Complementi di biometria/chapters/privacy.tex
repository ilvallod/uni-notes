\section{Privacy}

\paragraph{Continuous authentication}
Autenticazione biometrica non in modo istantaneo ma in modo continuo

\paragraph{Silent authentication}
L’utente non deve neanche accorgersi che sta facendo l’autenticazione. Esempio è l'utilizzo di smartwatch con il controllo dei movimenti

\paragraph{Classical password protection}
L’approccio tradizionale con password non è applicabile in quanto il dato biometrico cambia continuamente, diversamente è un replay attack. L’hash sarà sempre diverso

\subsection{Biometria cancellabile}
E' un modo per incorporare la protezione e le funzionalità sostitutive in biometria per creare un sistema più sicuro. Distorsione intenzionale e sistematicamente ripetibile delle caratteristiche biometriche

\paragraph{Feature transformation}
\begin{itemize}
    \item \textit{Bio-hashing}, anche detto \textit{salting}, la funzione di trasformazione è invertibile. Utilizzato per aggiungere un valore casuale al template biometrico prima di applicare la trasformazione con la chiave \textit{K}. In questo modo, anche se un attaccante dovesse compromettere il sistema e ottenere il template trasformato $F(T,K)$, non sarebbe in grado di risalire al template originale senza conoscere il valore del salting
    \item \textit{Non–invertible transformation}, una funzione che rende difficile dal punto di vista computazionare invertire un template trasformato in un template originale
\end{itemize}

\paragraph{Helper data} 
Template crittato memorizzato

\paragraph{Key binding}
Associare una chiave crittografica a una identità

\newpage