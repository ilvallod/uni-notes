\section{Privacy}

%Privacy in mobile apps and services
\subsection{Privacy in mobile apps and services}
Linking identity with sensitive data

\paragraph{Location Based Service}
Re-identification can also be based on frequent patterns of locations. Example: My LBS requests (for car sharing for example) originate from a specialised hospital every Tuesday. 

In a social network context co-location may become private information. Protection of location and absence privacy becomes trickier

%Privacy protection regulation and techniques
\subsection{Privacy protection regulation and techniques}

\paragraph{GDPR: EU General Data Protection Regulation}
\begin{itemize}
    \item \textit{Privacy by default}, default privacy settings should be the most protective
    \item \textit{Privacy by design}, new processes must be designed with data protection in mind
    \item \textit{Right to be forgotten}, individuals have the right of correcting or deleting their personal data
    \item \textit{Pseudonymization}, it separates sensitive data from the data respondents keeping the mapping between them accessible only to selected authorised entities 
    \item \textit{Data minimization}, acquire personal data only at the precision strictly necessary for the service
\end{itemize}

\paragraph{Security principles}
\begin{itemize}
    \item \textit{Confidentiality}, only authorized parties can access data
    \item \textit{Integrity}, data should not be altered without authorization
    \item \textit{Availability}
    \item \textit{Transparency}
    \item \textit{Unlinkability}, privacy-relevant data cannot be linked across domains. K-anonymity
    \item \textit{Intervenability}, data rectification or access
\end{itemize}

\paragraph{k-anonymity}
Each released record cannot be associated to less than $k$ possible respondents

\paragraph{Location k-anonymity}
The principle of k-anonymity requires that the individual must be indistinguishable from $k - 1$ other potential issuers of the LBS request. 

\paragraph{Fake location}
Simultaneous requests from $n+1$ locations (one of them being
the real one) and discard $n$ results

\paragraph{Spatial cloaking}
Spatial cloaking is an anonymization technique used in the context of k-anonymity. It involves enlarging the location of the requesting user to include other users' positions

The geographic position of the user needs to be enlarged to a cloaking region that includes $k-1$ other users before sending the request to the LBS. This region can be computed by a trusted anonymization service that knows many user positions, such as a mobile operator. Therefore, even in the worst-case scenario where an untrusted LBS provider can identify all $k$ users in the reported area, they can only determine that one of them searched for something, and there is only a chance of $\frac{1}{k}$ that this user was $x$.

\paragraph{Differential privacy}
A technique for data privacy protection used to safeguard the privacy of individuals participating in a data collection or analysis process. It involves adding statistical noise to the original data in order to mask personal information

\newpage