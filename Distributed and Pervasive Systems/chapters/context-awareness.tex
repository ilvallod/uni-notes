\section{Context awareness}
Traditional software applications cannot understand the context of a request, so users must make explicit requests with various parameters

\paragraph{Context}
 \textit{A series of circumstances or facts surrounding a particular event or situation}. In our case the event is a request for a mobile service

 \paragraph{Temporal context}
 Context history, enables deriving new context, predicting new context

 \paragraph{Adaptiveness}
Captures context data to automatically adapt behavior. Very important because there can be: changes in network connectivity; battery changes; changes in the environments
\begin{itemize}
    \item \textit{Adapt functionality}, change data flow, hide or expose functionality
    \item \textit{Adapt data}, more or less accurate/quality
\end{itemize}

\paragraph{Obtaining context}
\begin{itemize}
    \item \textit{Low level context}, directly acquired by sensors or explicit preferences indicated by the user in their profiles
    \item \textit{High level context}, obtained through inference methods on low-level context information. What the user is doing or the user’s \textit{mood} can be deduced from the activity that the user follows in the day. Knowing what the user is doing allows you to adapt the interface.
\end{itemize}

\paragraph{Context representation}
The same context information could be used by multiple applications and
even shared. A formal representation of the information is necessary to process it automatically

\begin{itemize}
    \item \textit{Flat models}, no historical data, no reasoning
    \item \textit{DB-Based models}, formal models such as ER or Context Modelling Language are used that allow a more effective reasoning on data
    \item \textit{Ontological models}, formal specification of a shared conceptualization
\end{itemize}

\newpage