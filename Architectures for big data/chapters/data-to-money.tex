\section{From big data to big money}
When architecting an IT solution, two fundamental aspects must be taken into account, \textit{benefit} and \textit{cost}.

%Cost
\subsection{Cost}

\paragraph{Capex}
\textit{Capital expenditure}, encloses funds spent on improving the company's assets, such as buildings, vehicles, machinery or software

\paragraph{Opex}
\textit{Operating expense} refers to the money spent on maintaining the effective functioning of a product or service, such as software license fees

\begin{center}
    \begin{tabular}{lcc} 
    \toprule
        EG & CAPEX & OPEX \\
    \midrule
        Buying software & X &  \\ 
        Buying hardware & X &  \\
        Software ad-hoc solution  & X &  \\
        Monthly software fee &   & X \\
        Cloud service &   & X \\
        Interview consultant for start-ups &   & X \\
    \bottomrule
   \end{tabular}
\end{center}

\paragraph{As a service}
$X$ as a service refers to something offered as a service, the complexity of which is hidden from the end customer. Some examples are:
\begin{itemize}
    \item \textit{Infrastructure as a service}: hardware offered by a certain provider
    \item \textit{Software as a service}: ability to use software without installing it
    \item \textit{Platform as a service}: ability to use complex platforms without worrying about complex configurations
\end{itemize}

%Team development
\subsection{Team development}
\paragraph{Project management} 
Process of bringing a team's work toward the completion of a certain goal by completing all requirements in a given period of time. Achieve all project goals given the constraints

\paragraph{GANTT chart} 
Graph with tasks and weeks, with their associations. Using this chart we can define \textit{how much} because uses a time plan so it's possible calculate the cost.

\paragraph{Waterfall} 
Development method where project phases are sequential, only once one part is finished can the next part begin. It is not flexible, you cannot change requirements in progress

The main problem of waterfall methodologies is assuming predictability. Software development is unpredictable and investing too much time upfront on a plan will inevitably generate waste

\paragraph{Agile} 
You divide the work into sprints, at each sprint the developers deliver a working version of the system.
High-level requirements, formalized in a \textit{Product Backlog}, are gathered before starting. There are several roles:
\begin{itemize}
    \item \textit{Product owner}: takes care of adding Product Backlog priorities. Each requirement in the Product Backlog becomes a \textit{User Story}, each sprint deals with completing a certain amount of it
    \item \textit{Scrum master}: takes care of checking that you don't break/follow the Agile model
    \item \textit{Team}: group of developers
    \item \textit{Stakeholders}: they evaluate the product
\end{itemize}

\newpage